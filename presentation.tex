\documentclass{beamer}

% Title Page Information
\title{WEB DEVELOPMENT}
\author{IDRISS BADO }
\date{}

\begin{document}

% Title Slide
\begin{frame}
    \titlepage
\end{frame}

% Section 1: General Awareness
\section{General Awareness}
\begin{frame}{Web Development Basics}
    \begin{itemize}
        \item \textbf{What is web development?}
        \begin{itemize}
            \item Web development is the process of creating websites or web applications.
            \item It is essential because it enables businesses and individuals to establish an online presence, share information, and provide services to users worldwide.
        \end{itemize}
    \end{itemize}
\end{frame}

\begin{frame}{HTML Knowledge}
    \begin{itemize}
        \item \textbf{Have you heard of HTML?}
        \item \textbf{What does HTML stand for?}
        \begin{itemize}
            \item HTML stands for \textbf{HyperText Markup Language}.
        \end{itemize}
        \item \textbf{Primary purpose in web development:}
        \begin{itemize}
            \item HTML is used to structure the content on a webpage, like adding headings, paragraphs, images, and links.
        \end{itemize}
    \end{itemize}
\end{frame}

% Section 2: Familiarity with CSS
\section{Familiarity with CSS}
\begin{frame}{CSS Understanding}
    \begin{itemize}
        \item \textbf{Are you familiar with CSS?}
        \item \textbf{What is CSS used for?}
        \begin{itemize}
            \item CSS (Cascading Style Sheets) is used to style and design the layout of web pages.
            \item It controls things like colors, fonts, spacing, and how the webpage looks on different screen sizes.
        \end{itemize}
    \end{itemize}
\end{frame}

% Section 3: JavaScript and Its Importance
\section{JavaScript and Its Importance}
\begin{frame}{Introduction to JavaScript}
    \begin{itemize}
        \item \textbf{Have you heard of JavaScript?}
        \item \textbf{What is JavaScript?}
        \begin{itemize}
            \item JavaScript is a programming language used to make websites interactive.
        \end{itemize}
        \item \textbf{Why is it crucial for building interactive websites?}
        \begin{itemize}
            \item It enables dynamic features like animations, real-time updates, and user interactions such as clicking buttons or submitting forms.
        \end{itemize}
    \end{itemize}
\end{frame}

% Section 4: Introduction to React and Node.js
\section{Introduction to React and Node.js}
\begin{frame}{React and Node.js Awareness}
    \begin{itemize}
        \item \textbf{Are you familiar with React and Node.js?}
        \item \textbf{React:}
        \begin{itemize}
            \item A JavaScript library used for building user interfaces, especially for creating single-page applications.
        \end{itemize}
        \item \textbf{Node.js:}
        \begin{itemize}
            \item A JavaScript runtime environment that allows you to build server-side applications.
        \end{itemize}
    \end{itemize}
\end{frame}

% Section 5: Bonus Question
\section{Bonus Question}
\begin{frame}{Learning Intentions}
    \begin{itemize}
        \item \textbf{Why do you want to learn web development?}
        \begin{itemize}
            \item To build functional and visually appealing websites.
            \item To create interactive web applications that solve real-world problems.
        \end{itemize}
        \item \textbf{What do you hope to achieve?}
        \begin{itemize}
            \item Gain proficiency in HTML, CSS, JavaScript, React, and Node.js.
            \item Develop skills to pursue a career in web development or enhance your existing knowledge.
        \end{itemize}
    \end{itemize}
\end{frame}

\end{document}

